\documentclass[12pt]{article}
\title{ General Principles of Radiation Protection in Fields of Diagnostic Medical Exposure}
\author{UPDATE 2}
\setlength{\parindent}{4em}
\setlength{\parskip}{1em}\usepackage{datetime}
\date{28th January 2022}
\begin{document}
\maketitle
In publication 26, the International Commission on Radiological Protection (ICRP) suggested a system of radiation protection based on the three concepts of justification, optimization, and individual dose limitation proposed by the ICRP. The ICRP changed its guidelines in publication 60 and expanded its concept to include a radiological protection system while maintaining the core principles of protection. In 2007, the International Commission on Radiological Protection (ICRP) issued Report 103, which was a revised general proposal for a radiation protection system. The new recommendations offer guidance on the essential concepts that may be used to develop suitable radiation protection.Previous ICRP Recommendations emphasised the importance of principles of protection in the protection system, and they have now created a single set of principles that apply to planned, emergency, and existing exposure circumstances. They further highlighted how the fundamental concepts apply to radiation sources and individuals, as well as how source-related rules apply to all controlled scenarios in these Recommendations.
Two source-related concepts apply in all exposure circumstances.
1. The justification principle: Any choice that affects radiation exposure should cause more good than damage. This means that by adding a new radiation source, lowering present exposure, or reducing the danger of future exposure, one should be able to obtain enough individual or societal benefit to compensate for the harm it causes.
2. The concept of protection optimization: the chance of exposure, the number of persons exposed, and the volume of their individual doses should all be kept as low as practically possible, taking economic and social variables into account.
\end{document}
