\documentclass[12pt]{article}
\title{General Principles of Radiation Protection in Fields of Diagnostic Medical Exposure}
\author{UPDATE 11}
\setlength{\parindent}{4em}
\setlength{\parskip}{1em}\usepackage{datetime}
\date{1st April 2022}
\begin{document}
\maketitle

\textbf{Summary & Conclusions:}
Radiation protection is an essential part of any radiology department's operational infrastructure. The main principles of radiation protection are to protect personnel directly or indirectly involved with radiation from undue radiation exposure while not limiting the benefits of radiation exposure. Justification of the procedure involving radiation exposure, use of a minimum radiation exposure compatible with the procedure that provides adequate diagnostic information, shielding of personnel and patients from unwanted radiation exposures, and monitoring of radiation exposure to occupational workers and the working environment are some of the components of radiation protection.

\raggedright The RSO and other administrative authorities of the department/hospital are responsible for regular surveillance of the department for radiation levels, monitoring of radiation protection programmes, and regular educational activities. In these surveys and protection programmes, the ICRP and AERB standards must be followed.

\end{document}
