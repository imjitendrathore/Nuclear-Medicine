\documentclass[12pt]{article}
\title{General Principles of Radiation Protection in Fields of Diagnostic Medical Exposure}
\author{UPDATE 7}
\setlength{\parindent}{4em}
\setlength{\parskip}{1em}\usepackage{datetime}
\date{4th March 2022}
\begin{document}
\maketitle

\textbf{Optimization:}
It's also one-of-a-kind to optimise patient safety. Individual constraints on patient dosage might be contradictory to the medical objective of the process in optimising patient protection in diagnostic procedures, as the benefit and risk are shared by the same person. 

In medicine, radiation protection for patients is normally optimised on two levels: 
(1) the design, suitable selection, and construction of equipment and installations; and (2) the day-to-day working procedures. 

\raggedright The primary goal of this radiation protection optimization is to alter the protective mechanisms for a source of radiation to maximise the net benefit. The optimization of protection in medical exposures does not always imply a reduction in patient dosages.

The term "optimization of radiological protection" refers to maintaining doses "as low as reasonably practicable, taking economic and societal concerns into account," and is best summarised as managing the radiation dosage to the patient to be commensurate with the medical goal.
\end{document}
