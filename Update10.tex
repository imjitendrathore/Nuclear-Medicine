\documentclass[12pt]{article}
\title{General Principles of Radiation Protection in Fields of Diagnostic Medical Exposure}
\author{UPDATE 10}
\setlength{\parindent}{4em}
\setlength{\parskip}{1em}\usepackage{datetime}
\date{25th March 2022}
\begin{document}
\maketitle

\textbf{Radiation protection survey and programme:}
The hospital administration/owners of the X-ray facility are responsible for establishing a radiation protection programme. A Radiation Safety Committee (RSC) and a Radiation Safety Officer are expected to be appointed by the administration (RSO). The RSC should include a radiologist, a medical physicist, a nuclear medicine specialist, a senior nurse, and an internist, according to NCRP. RSC is responsible for conducting regular radiation protection surveys. There are five phases to this survey:

\raggedright 1. Research: To learn more about the department's layout, workload, personnel monitoring, and records.

2. Inspection: Each diagnostic installation in the department is inspected for its protection status in terms of operating factors, control booth, and protection device availability.

3. Measurement: Measurement are carried out on exposure factors. In radiography and fluoroscopy, scattered radiation and patient dose measurements are also performed.

4. Evaluation: Records, equipment operation, protective clothing status, and radiation doses obtained from phase-3 are used to assess the department's radiation protection status.

5. Recommendations: A report is prepared on the department's protection status and, if any, identified problem areas, with recommendations for corrective measures.
\end{document}
