\documentclass[12pt]{article}
\title{General Principles of Radiation Protection in Fields of Diagnostic Medical Exposure}
\author{UPDATE 9}
\setlength{\parindent}{4em}
\setlength{\parskip}{1em}\usepackage{datetime}
\date{18th March 2022}
\begin{document}
\maketitle

\textbf{Radiation detection and measurement:}
Radiation detection devices are the instruments that are used to detect radiation. Radiation dosimeters are instruments that measure radiation.

\raggedright Radiation detection can be done in a variety of ways, all of which are based on the physical and chemical effects of radiation exposure. These are the methods:

Ionization: The ability of radiation to produce ionisation in air is the foundation for the ionisation chamber's radiation detection. It consists of an electrode placed in the middle of a gas-filled cylinder. When x-rays enter the chamber, they ionise the gas, resulting in the formation of negative ions (electrons) and positive ions (protons) (positrons). The positively charged rod collects the electrons, while the positively charged ions are attracted to the cylinder's negatively charged wall. The small current generated by the chamber is then amplified and measured. The current's strength is proportional to the intensity of the radiation.

The photographic effect: The photographic effect is the foundation of detectors that use film. It refers to the ability of radiation to blacken photographic films.

Luminescence: It is the property of certain materials that causes them to emit light when they are stimulated by a physiological process, a chemical or electrical action, or heat. The electrons in these materials are raised to higher orbital levels when they are struck by radiation. Light is emitted when they return to their original orbital level. The radiation intensity is proportional to the amount of light emitted. When heat is applied to lithium fluoride, for example, it produces light. Thermoluminescence dosimetry (TLD), a method for measuring exposure to patients and personnel, is based on this principle.

Scintillation: A flash of light is referred to as scintillation. Certain crystals, such as sodium iodide and cesium iodide, have the ability to absorb radiation and convert it to light. After that, the light is directed to a photomultiplier tube, which converts it into an electrical pulse. The pulse's size is proportional to the light intensity, which is proportional to the radiation's energy.
\end{document}
