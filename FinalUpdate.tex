\documentclass{article}
\usepackage[utf8]{inputenc}
\usepackage{graphics}
\usepackage{multicol}
\usepackage{pgfplots}
\usepackage{float}
\usepackage{biblatex} %Imports biblatex package
\addbibresource{name.bib} %import the bibliography file
\usepackage[paper=a4 paper, left=3cm, right=3cm, top=2cm]{geometry}

\title{\textbf{General Principles of Radiation Protection in fields of Diagnostic Medical Exposure}}
\author{Under the guidance of Dr. Saurabh Gupta\\
Submitted by Jitendra Rathore(Roll No. - 18111028) \\ \testit{Department of Biomedical Engineering} \\National Institute of Technology, Raipur }
\date{April 2022}

\begin{document}

\maketitle
\large
\begin{Abstract}\textbf{Abstract: Radioactive doses through therapeutic exposure seem to be the most common type of man-made radiation exposure, because to the fast advancement of clinical technology such as CT or PET-CT. Three major phrases encapsulate the theoretical foundations of radiation protection against the dangers of ionizing radiation: justification, optimization, and dosage limit. Diagnostic reference levels, rather than dosage limits, were commonly employed as a reference value since therapeutic exposure to radiation involves special implications. At the regional and international levels, there seem to be a number of proposed organizations that set radiation safety standards. the National Commission for Radiation Protection (NCRP) in the United States, The International Commission for Radiation Protection (ICRP), , and the Atomic Energy Regulatory Board (AERB) in India are the organizations in charge. Clinicians and radiologist have to be conscious of the hazards and advantages of medical radiation, as well as comprehend and apply radiation protection protocols for patient. Referring physicians and radiologists must also be educated.} 
\end{Abstract}
\section{Introduction}

Healthcare radiation has been frequently utilized since Wilhelm Conrad Roentgen's invention of X-rays, and it really is unavoidable in the diagnosis and treatment of patients. The globe shuddered in dread of radiation after the March 2011 nuclear disaster in Japan. As a result, there has been a rise in awareness of therapeutic exposure to radiation and awareness in radiation shielding in Korea. 
The International Commission on Radiation Protection (ICRP) proposed three essential phrases to describe generic radiation safety principles: justification, optimization, and dosage limit. "The overarching purpose of radiation protection," the ICRP declared in 1991, "is to establish an acceptable level of protection for man without excessively restricting the beneficial behaviors that give rise to radiation exposure." "Current standards of protection are aimed to avoid the occurrence of deterministic effects by keeping doses below relevant thresholds and guarantee that all possible precautions are taken to limit induction of stochastic effects," the ICRP added. Dosage limits or dose limitations are not acceptable since medical exposure of patients has special concerns. Dose limitations are irrelevant since ionizing radiation, when utilized at the proper dose for the specific medical purpose, is a vital instrument that will do more good than damage. \\


As a result, medical radiation has no dosage restrictions, and the diagnostic reference level (DRL) is commonly employed as a reference value. To reduce the hazards of radiation exposure, any therapeutic radiation exposure must be validated, and ionizing radiation exams should be optimized. The examination must be medically recommended and beneficial in order to be justified. The term "optimization" refers to the use of dosages that are as low as reasonably attainable (ALARA) in relation to the diagnosis job.


\section{Principles and Guidelines}

The following are fundamental radiation protective concepts based on ICRP guideline 103 (2): In publication 26, the International Commission on Radiological Protection (ICRP) presented a radiation protection system depending on 3 principles: justification, optimization, and individual dose limiting. The ICRP changed its guidelines in publication 60 and expanded its concept to include a radioactive protection system while maintaining the core principles of protection. In 2007, the International Commission on Radiological Protection (ICRP) issued Report 103, which was a modified fundamental proposal for a radiation protection system. These new suggestions offer guidance on the essential concepts that may be used to develop suitable radiation protection.

\subsection{The principles of radioactive protection:} The ICRP has created a particular set of principles that apply to anticipated, emergencies, and current exposure circumstances, as outlined in earlier Guidelines. They further highlighted how the basic concepts applied to radiation sources and individuals, and how source-related rules apply to all controlled scenarios in these Guidelines. In all exposure circumstances, there are 2 source-related guidelines to follow:\\

\textbf{1.	The justification principle:} Any choice which affects exposure to radiation must cause greater benefit than damage. This implies that just by adding an additional radiation source, lowering present exposure, or decreasing the danger of future exposure, one ought to be able to obtain enough personal or societal benefits to compensate for the damage it causes.\\

\textbf{2.	The principle of protection optimization:} the chance of exposure, the number of people affected, as well as the volume of their dosages should be maintained as lower as practically possible, considering social and economic variables into consideration. This indicates that the amount of safety ought to be the highest possible given the conditions, optimising the benefit-to-harm ratio. To prevent extremely unequal results from this optimization approach, dosages or dangers to individuals from a given source must be limited (dose or risk constraints and reference levels). One principle is personal and applies to circumstances where exposure is planned. \\

\textbf{3.	The principle of dosage limit application:} In planned exposure circumstances other than medical exposure of , the overall dosage from authorized source to any individual must not surpass the acceptable limits specified by the Commission. To limit individual doses, the ideas of dosage limitation and reference level are employed in combination with the optimization of protection. Individual dosage levels ought be established, whether as a dose limitation or a reference level. The first goal would be to not exceed or stay below these levels, with the ultimate goal being to lower all dosages to as low as reasonably practicable, taking into consideration economic and sociological concerns. Diagnostic reference levels are often used in clinical diagnosis (i.e., planned exposure settings) to show if the level of patient’s dosage or administered activity from a certain imaging method are exceptionally low or high for such operation under normal circumstances. If this is the case, a local evaluation should be conducted to see if protection has indeed been effectively optimized or if any preventative action is necessary.


\subsection{Categories of exposure: }

The Commission divides exposures into three categories: occupational exposures, public exposures, and patient medical exposures. \\

\textbf{1.	Occupational exposure:} Occupational exposure refers to all radiation that employees are exposed to as a consequence of their jobs. The ICRP uses the term "occupational exposures" to describe radiation exposures that occur at work as a result of events that may reasonably be attributed to operating management. The employer bears the primary duty for worker safety. \\

\textbf{2.	Public exposure:} Except for workplace exposures and patient medical exposures, public exposure refers to all public exposures. It can be caused by a variety of radiation sources. Although natural sources account for the majority of public exposure, this does not justify a reduction in the attention devoted to lesser, but more easily manageable, exposures from man-made sources. Pregnant employees' embryos and foetuses are treated as public exposures and are controlled accordingly. \\

\textbf{3.	Medical exposure of patients:} In diagnostic, interventional, and therapeutic treatments, patients are exposed to radiation. Several aspects of radiological practices in healthcare necessitate a different strategy than radiation protection in other intentional exposure circumstances. The exposure is done on purpose and for the patient's benefit. As a result, applying these Guidelines to medical applications of radiation necessitates independent advice.

\section{ICRP Recommendation 105 (1): Principles of Radiation 
Protection in Medical Fields}

Various aspects of exposure to radiation in healthcare for patients necessitate a distinct strategy to radiation safety than other forms of exposure to radiation. Patients are purposefully exposed. Except for in radiation treatment, the goal is to use the radiation to provide diagnostic information or to perform an interventional operation rather than to administer a radiation dosage. Medical applications of radiation for patients are purely voluntary, with the expectation of a significant benefit to the patient's condition. The voluntary decision is made with varied degrees of informed consent, which takes into account both the expected benefit and the potential hazards. The quantity of data presented to acquire informed consent varies depending on the level of exposure (diagnostic, interventional, or therapeutic) and the potential for unexpected medical consequences caused by radiation exposure. Use of radiation in biomedical research is the exception to the idea of a voluntary exposure resulting to a direct individual medicinal value. In such cases, the voluntary exposure is typically for the good of society rather than for the benefit of the person. Consent must always be given with knowledge. The goal of screening is to discover a disease process which has not yet manifested medically. Modern ionizing radiation screening procedures tend to be effective and are suggested for specific groups. Patients must be thoroughly informed about the possible advantages and hazards of screening, especially radiation concerns. Each use of ionizing radiation for screening asymptomatic people should be assessed and justified in terms of clinical utility. \\

\textbf{Application of radiation protection principles in areas of medicine:} Since clinical exposure of individuals involves special concerns, it covers the correct application of radiation protection's essential principles (justification, optimization of protection, and application of dose limits). Dose restrictions are not acceptable when it comes to medical exposure of patients since they typically do more damage than benefit. There are frequently other medical issues that are more serious, severe, or even life-threatening than the radiation exposure. The reason for medical operations is emphasized, as well as the optimization of radiation protection. The acceptable techniques to minimize unwanted or wasteful radiation exposure in diagnostic and therapeutic procedures include justification of procedures (for a specified goal and for a specific patient) and management of the patient dose commensurate with the medical job. The most successful measures are likely to be equipment characteristics that assist patient dose control and diagnostic reference values produced at the appropriate national, regional, or local level. The prevention of mishaps is a major concern in radiation treatment. Dose limits are acceptable for comforters and caretakers, as well as volunteers in biomedical research.

\subsection{Justification:  }

The justification for radiological patient protection differs from the justification for other radiation applications in that the advantages and dangers involved with a procedure are usually experienced by the same individual. (There may be additional factors to consider attendant occupational exposures may be linked to patient dosages, or there may be a trade-off; screening programmes may help the entire population rather than each individual who is examined.) However, in most cases, the risks and profits are shared by the same person). And, perhaps most importantly in everyday medical practice, just because a treatment or procedure is permissible in general does not indicate that this is justified in the context of the individual patient under consideration. In medicine, there are three degrees of justification for a radiographic practice. \\

1.	At the most basic level, it is widely understood that the right use of radiation in medicine benefits society more than it harms it.

2.	A specific technique with a specific goal is established and justified at the second level. The purpose of the second level of justification is to determine if the radiological technique would enhance the diagnosis or treatment of the exposed persons, or will offer required information about them.

3.	At the third level, the procedure's applicability to a specific patient must be substantiated. As a result, all individual medical exposures should be justified in advance, taking into consideration the exposure's unique aims as well as the individual's characteristics.

\subsection{Optimization:}
\large
It's also one-of-a-kind to optimize patient safety. Individual constraints on patient dosage might be contradictory to the medical objective of the process in optimizing patient protection in diagnostic procedures, as the advantage and hazard are shared by the very same person. In healthcare, radiological safety for patients is normally optimized at two levels:
(1) the design, suitable selection, and construction of equipment and installations; and (2) day-to-day working procedures. \\

The primary goal of this radiation safety optimization is to alter the protective mechanisms for a source of radiation to maximize the net benefit. The optimization of protection in medical exposures does not always imply a reduction in patient dosages. The term "optimization of radiological protection" refers to the management of the radiation dosage to the patient to be commensurate with the therapeutic aim while maintaining doses "as low as reasonably practicable, taking economic and social aspects into account."

\subsection{Diagnostic reference levels:}

The diagnostic reference level (DRL) is a type of inquiry level that pertains to therapeutic exposure. DRLs are meant to support clinical judgement, not to draw a boundary among "good" and "poor" therapy. They aid in the practise of radiology in healthcare. The numerical values of DRLs are only suggestions; nevertheless, an authorised authority may compel that the DRL idea be implemented. The numeric numbers for DRLs should not be used as regulatory limitations or for commercial reasons. The parameters must be checked at times that strike a balance among needed stability and long-term changes in the reported dosage ranges. The parameters chosen might be unique to a country or area. By lowering the occurrence of unreasonable lower or higher outcomes, a DRL may be utilised to enhance a regional, national, or local distribution of observed outcomes for a general medical imaging job. It also encourages the achievement of a smaller and more appropriate range of values that indicate excellent practise for certain imaging methods. \\

The following are the guiding factors for determining a DRL:\\

1.	A clearly defined objective is specified at the regional, national, or local level, including degree of clinical and technical specifications for the medical imaging task

2.	Based on relevant regional, national, or local data, the DRL value is selected. 

3.	It is possible to obtain a practical quantity for the DRL. 

4.	Accordingly, DRL can be used to measure the relative change in tissue doses of patients and, therefore, the relative change in risk for a given medical imaging procedure. 

5.	A clear illustration of how the DRL should be used in practice is provided.

\section{Current status of medical radiation exposure}

\large Radiation doses from CT are currently the single biggest source of diagnostic radiation exposure to patients, thanks to the fast adoption of multi-detector CT. The increase in medical exposure was the primary source of the large shift in ionizing radiation exposure; other factors did not alter much. The United States population's ionizing radiation exposure has changed, according to the National Council on Radiation Protection and Measurements. In the 1980s, the estimated average radiation dose in the United States was 3.6 mSv, but by 2006, it had risen to 6.2 mSv. Between the 1980s and 2006, medical exposure accounted for a 72 percent rise in ionizing radiation exposure, going from 15 percent of all exposure in the 1980s to 48 percent of all exposure in 2006. CT was the most significant source, accounting for 49 percent of all medical exposure. Nuclear medicine is responsible for 26\% of all medical exposure. The results did not include radiation therapy-related exposure. Kim looked at the number of diagnostic radiation procedures in Korea using National Insurance statistics. Between 2007 and 2011, the total number of photos rose by 37\%. Conventional radiography received 78\% of the vote, whereas dental radiography received 11\%, mammography received 7\%, CT received 3\%, fluoroscopy received 1\%, and angiography received 0.2 percent. Based on survey data from Korea, they calculated effective dosages for CT and traditional radiography. Despite significant limitations, the projected total dosage of medical diagnostic exams to the Korean population was 68,000 manSv. \\

\large Since 2007, the projected collective dosage has grown by 50\%. The annual effective dosage per person was 1.4 mSv, with CT contributing 0.79 mSv, conventional radiography 0.44 mSv, fluoroscopy 0.09 mSv, angiography 0.05 mSv, mammography 0.02 mSv, and dental exams 0.004 mSv. Controlling medical radiation exposure necessitates regulation. Korea has two independent control systems for medical radiation exposure. First, the Medical Services Act, which is administered by the Ministry of Health and Welfare, governs diagnostic radiation regulation. Therapeutic radiation and nuclear medicine, on the other hand, are governed by the Nuclear Safety Act, which is enforced by the Nuclear Safety and Security Commission. The medical service act has two parts: one is for the safety control of diagnostic X-ray generating equipment; management of equipment and radiation protection facilities, which is regulated by the Korean Center for Disease Control, and the other is for the installation and operation of special medical equipment quality assurance systems on CT, MRI, and mammography, which is regulated by the Korean Center for Disease Control. These regulations include equipment management, occupational exposure, and radiation protection facilities, but they do not address patient safety. Many national and international organizations have worked hard to establish standards, accreditation, and even legislation to control and monitor medical radiation exposure. Continuous involvement from health experts and organizations might help achieve this. 

\section{Conclusion}

\large Radiation shielding is an essential feature of every radiology department's operational infrastructure. The primary concepts of radiation protection are to safeguard persons explicitly or implicitly working with radiation from excessive radiation exposure while not restricting the advantages of exposure to radiation. Justification of the procedure involving radiation exposure, use of a least radiation exposure suitable with the procedure that provides sufficient diagnostic data, shielding of personnel and patients from undesired radioactive materials, and tracking of exposure to radiation to occupational workers and the working environment are some of the components of radiation protection. The Radiation Safety Officer and other administrative authorities of the department/hospital are responsible for frequent surveillance of the department for radiation levels, monitoring of radiation protection programmes, and regular teaching activities. Physicians and radiologists must be aware of the hazards and benefits of medical radiation exposure, as well as comprehend and apply radiation protection protocols for patients. The referring physician's and radiologist's education is also crucial.

\newpage 
\Large\textbf{Reference} \\

\large
1.	Antoni, Rodolphe, and Laurent Bourgois. 2017. “Protection Against External Exposition, General Principles.” Biological and Medical Physics, Biomedical Engineering. https://doi.org/10.1007/978-3-319-48660-4_5. \\

2.	Committee for Review and Evaluation of the Medical Use Program of the Nuclear Regulatory Commission, and Institute of Medicine. 1996. Radiation in Medicine: A Need for Regulatory Reform. National Academies Press. \\

3.	Do, Kyung-Hyun. 2016. “General Principles of Radiation Protection in Fields of Diagnostic Medical Exposure.” Journal of Korean Medical Science 31 Suppl 1 (February): S6–9. \\

4.	Domenech, Haydee. 2017. “General Principles of Radiation Protection.” Radiation Safety. https://doi.org/10.1007/978-3-319-42671-6_11. \\

5.	International Commission on Radiological Protection. Radiological protection in medicine. ICRP Publication 105. Ann ICRP 2007; 37: 1-63. \\

6.	International Commission on Radiological Protection. The 2007 Recommendations of the International Commission on Radiological Protection. ICRP publication 103. Ann ICRP 2007; 37: 1-332. \\

7.	Mettler FA Jr, Bhargavan M, Faulkner K, Gilley DB, Gray JE, Ibbott GS, Lipoti JA, Mahesh M, McCrohan JL, Stabin MG, et al. Radiologic and nuclear medicine studies in the United States and worldwide: frequency, radiation dose, and comparison with other radiation sources--1950-2007. Radiology 2009; 253: 520-31 \\

8.	National Council on Radiation Protection and Measurements. Ionizing radiation exposure of the population of the United States: NCRP report no. 160. Bethesda, MD: National Council on Radiation Protection and Measurements, 2009. \\

9.	Kim KP. Radiation exposure of Korean population from medical diagnostic examinations. Seoul: Ministry of Food and Drug Safety, 2013


\end{document}
