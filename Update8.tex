\documentclass[12pt]{article}
\title{General Principles of Radiation Protection in Fields of Diagnostic Medical Exposure}
\author{UPDATE 8}
\setlength{\parindent}{4em}
\setlength{\parskip}{1em}\usepackage{datetime}
\date{11th March 2022}
\begin{document}
\maketitle

\textbf{Diagnostic reference levels:}
The diagnostic reference level (DRL) is a type of investigation level that applies to medical exposure. DRLs are meant to supplement professional judgement, not to draw a line between "good" and "bad" medicine. They aid in the practise of radiology in medicine. The numerical values of DRLs are merely suggestions; however, an authorised body may require that the DRL concept be implemented. The numerical values for DRLs should not be used as regulatory limits or for commercial purposes. The values should be checked at intervals that strike a balance between required stability and long-term changes in observed dose distributions. The chosen values may be unique to a country or region.By reducing the frequency of unjustified high or low values, a DRL can be used to improve the regional, national, or local distribution of observed results for a general medical imaging task. It also encourages the development of a more narrow and optimal range of values that represent best practise for specific imaging protocols.

\raggedright 

The following are the guiding principles for establishing a DRL: 1 the regional, national, or local objective is clearly defined, including the degree of specification of clinical and technical conditions for the medical imaging task; 2 the DRL value chosen is based on relevant regional, national, or local data; 3 the quantity used for the DRL can be obtained in a practical manner; 4 the quantity used for the DRL is a suitable measure of the relative change in patient tissue dosimetry; 5. A clear illustration of how the DRL should be used in practise is provided.
\end{document}
