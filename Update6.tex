\documentclass[12pt]{article}
\title{General Principles of Radiation Protection in Fields of Diagnostic Medical Exposure}
\author{UPDATE 6}
\setlength{\parindent}{4em}
\setlength{\parskip}{1em}\usepackage{datetime}
\date{25th February 2022}
\begin{document}
\maketitle

\textbf{Justification:}
The justification for radiological patient protection differs from the justification for other radiation applications in that the advantages and dangers associated with a procedure are usually experienced by the same individual. (There may be additional factors to consider: attendant occupational exposures may be linked to patient dosages, or there may be a trade-off; screening programmes may help the entire population rather than each individual who is examined.) However, in most cases, the risks and profits are shared by the same person). And, perhaps most importantly in everyday medical practise, just because a treatment or procedure is permissible in general does not indicate that it is justified in the context of the individual patient under consideration. 

In medicine, there are three degrees of justification for a radiographic practise.
1. At the most basic level, it is widely agreed that the correct use of radiation in medicine benefits society more than it harms it.
2. A specific technique with a specific goal is established and justified at the second level. The purpose of the second level of justification is to determine if the radiological technique would enhance the diagnosis or treatment of the exposed persons, or will offer required information about them.
3. At the third level, the procedure's applicability to a specific patient must be justified. As a result, all individual medical exposures should be justified in advance, taking into consideration the exposure's unique aims as well as the individual's characteristics.

\end{document}
