\documentclass[12pt]{article}
\title{ General Principles of Radiation Protection in Fields of Diagnostic Medical Exposure}
\author{UPDATE 4}
\setlength{\parindent}{4em}
\setlength{\parskip}{1em}\usepackage{datetime}
\date{11th February 2022}
\begin{document}
\maketitle

Principles of radiation protection in medical fields from the ICRP Recommendation 105: In medicine, radiation protection has its own set of properties.Several aspects of radiation exposure in medicine for patients necessitate a distinct strategy to radiological protection than for other forms of radiation exposure. Patients are purposefully exposed. Except in radiation treatment, the goal is to use the radiation to provide diagnostic information or to perform an interventional operation rather than to administer a radiation dosage. Medical applications of radiation for patients are purely voluntary, with the expectation of a direct benefit to the patient's health. The voluntary decision is made with varied degrees of informed consent, which takes into account both the expected benefit and the potential hazards. The quantity of information presented to acquire informed consent varies depending on the level of exposure (diagnostic, interventional, or therapeutic) and the potential for unexpected medical consequences due to radiation exposure. The use of radiation in biomedical research is an exception to the idea of a voluntary exposure leading to a direct individual medical benefit. In these cases, the voluntary exposure is typically for the good of society rather than for the benefit of the person. Consent must always be given with knowledge. The goal of screening is to discover a disease process that has not yet manifested clinically. Current ionising radiation screening procedures tend to be effective and are suggested for specific groups.Patients should be thoroughly informed about the possible advantages and hazards of screening, including radiation concerns. Each use of ionising radiation for screening asymptomatic people should be assessed and justified in terms of its clinical utility.

\end{document}
