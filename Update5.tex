\documentclass[12pt]{article}
\title{ General Principles of Radiation Protection in Fields of Diagnostic Medical Exposure}
\author{UPDATE 5}
\setlength{\parindent}{4em}
\setlength{\parskip}{1em}\usepackage{datetime}
\date{18th February 2022}
\begin{document}
\maketitle

\textbf{Application of principles of radiation protection in medical fields:}
The efficient application of the essential concepts (justification, optimization of protection, and application of dose limits) of radiation protection is addressed in medical domains since medical exposure of patients has unique considerations. Dose restrictions aren't acceptable when it comes to medical exposure of patients because they typically do more damage than benefit. Radiation exposure is frequently accompanied by persistent, severe, or even life-threatening illness disorders. The focus is therefore on the medical operations' rationale as well as the radiological protection's optimization.

The acceptable techniques to minimise unnecessary or unproductive radiation exposure in diagnostic and interventional procedures include justification of procedures (for a specified goal and for a specific patient) and management of the patient dose commensurate with the medical job. The most successful measures are likely to be equipment characteristics that assist patient dose control and diagnostic reference values produced at the appropriate national, regional, or local level. The prevention of mishaps is a major concern in radiation treatment. Dose limits are acceptable for comforters and caretakers, as well as volunteers in biomedical research.

\end{document}
