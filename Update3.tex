\documentclass[12pt]{article}
\title{ General Principles of Radiation Protection in Fields of Diagnostic Medical Exposure}
\author{UPDATE 3}
\setlength{\parindent}{4em}
\setlength{\parskip}{1em}\usepackage{datetime}
\date{04th February 2022}
\begin{document}
\maketitle
The Commission divides exposures into three categories: occupational exposures, public exposures, and  Medical exposure of patients.

1. Occupational exposure: Occupational exposure refers to all radiation that employees are exposed to as a result of their employment. Radiation exposures incurred at work as a result of conditions that may fairly be regarded as being the responsibility of the operational management are limited by the ICRP. The employer bears the primary duty for worker safety.

2. Public exposure: Except for occupational exposures and patient medical exposures, public exposure refers to all public exposures. It can be caused by a variety of radiation sources.Although natural sources account for the majority of public exposure, this does not justify a reduction in the attention devoted to lesser, but more easily manageable, exposures from man-made sources. Pregnant employees' embryos and foetuses are evaluated and controlled as public exposures.

3. Medical exposure of patients: In diagnostic, interventional, and therapeutic treatments, patients are exposed to radiation. Several aspects of radiological practices in medicine necessitate a different strategy than radiation protection in other intentional exposure circumstances. The exposure is done on purpose and for the patient's benefit. As a result, applying these Recommendations to medical applications of radiation necessitates independent advice.

\end{document}
